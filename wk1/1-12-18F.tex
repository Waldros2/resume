\documentclass{article}%
\usepackage{amsmath}%
\usepackage{amsfonts}%
\usepackage{amssymb}%
\usepackage{graphicx}%
\usepackage{multicol}
%-------------------------------------------
\newtheorem{theorem}{Theorem}
\newtheorem{acknowledgement}[theorem]{Acknowledgement}
\newtheorem{algorithm}[theorem]{Algorithm}
\newtheorem{axiom}[theorem]{Axiom}
\newtheorem{case}[theorem]{Case}
\newtheorem{claim}[theorem]{Claim}
\newtheorem{conclusion}[theorem]{Conclusion}
\newtheorem{condition}[theorem]{Condition}
\newtheorem{conjecture}[theorem]{Conjecture}
\newtheorem{corollary}[theorem]{Corollary}
\newtheorem{criterion}[theorem]{Criterion}
\newtheorem{definition}[theorem]{Definition}
\newtheorem{example}[theorem]{Example}
\newtheorem{exercise}[theorem]{Exercise}
\newtheorem{lemma}[theorem]{Lemma}
\newtheorem{notation}[theorem]{Notation}
\newtheorem{problem}[theorem]{Problem}
\newtheorem{proposition}[theorem]{Proposition}
\newtheorem{remark}[theorem]{Remark}
\newtheorem{solution}[theorem]{Solution}
\newtheorem{summary}[theorem]{Summary}
\newenvironment{proof}[1][Proof]{\textbf{#1.} }{\ \rule{0.5em}{0.5em}}
\setlength{\textwidth}{7.0in}
\setlength{\oddsidemargin}{-0.35in}
\setlength{\topmargin}{-0.5in}
\setlength{\textheight}{9.0in}
\setlength{\parindent}{0.3in}
\begin{document}

\begin{flushright}
\textbf{Scott Waldron \\
        Winter 2018}
\end{flushright}

\begin{flushleft}
\textbf{CS330 \\
Lecture Notes} \\
\end{flushleft}
\begin{center}
1-12-18
\end{center}

What does a simple database contain?

users $\longleftrightarrow$ program $\longleftrightarrow$ Database\\
\begin{itemize}
\item Bottom Part (Disk Storage)
\begin{itemize}
\item Data is stored here
\item Indices
\item Data Dictionary (Metadata)
\item Statistical Data
\item DBMS hides details of how data is stored and maintained (Data abstraction)
\begin{itemize}
\item Why is it important?\\
Not to overwhelm the user, security and concurrency.
\end{itemize}
\end{itemize}
\item Middle Part of DBMS
\begin{itemize}
\item Query processor helps DBMS to simplify and facilitate access to data
\begin{itemize}
\item Query: A statement requesting information
\item Queries are represented by a language (Database language)
\item There are two parts: DDL (Data definition language), DML (Data manipulation language)
\end{itemize}
\item Storage Manager is important because DB typically requires a large storage space
\begin{itemize}
\item Buffer Manager
\begin{itemize}
\item Fetch data from disk storage into main memory
\item Decide what data to cache into main memory
\end{itemize}
\item File Manager
\begin{itemize}
\item Manages space allocation onn disk storages
\end{itemize}
\item Authorization and integrity constraints
\begin{itemize}
\item Tests for satisfaction of integrity constraints
\item Checks authority of users
\end{itemize}
\item Transaction Manager
\begin{itemize}
\item A unit of program that accesses and updates data items
\item Who initiates a transaction? SQL or programming language using ODBC/JDBC
\item What does transaction manager do? Ensures ACID properities\\
(Atomicity, Consistency, Isolation, Durability)
\item \textbf{Atomicty} (All or none transaction)
\item \textbf{Consistency} (Preserves consistency of DB)
\item \textbf{Durability} (After successful funds transfer, new values of A and B must persist, even if system fails)
\item \textbf{Isolation} (For two transactions $T_i$ $T_j$, it appears to $T_i$ that either $T_j$ finished execution before $T_i$ started or $T_j$ started execution after $T_i$ finished)
\pagebreak
\end{itemize}
\end{itemize}
\end{itemize}
\item DBMS solves these 9 problems
\begin{enumerate}
\item More complexity
\item Data Redundancy
\item Data Inconsistency
\item Difficulty in accessing data
\item Data Isolation
\item Integrity Problem
\item Atomicity problems
\item Concurrent-access anomalies
\item Security Problems
\end{enumerate}
\end{itemize}
\end{document}
